\chapter{公式}

这里直接给出几个较为复杂的公式的例子,可一一进行参照。
若有未包含的数学符号或公式格式,请参阅本模板所包含的手册(本地manual文件夹)或百度必应谷歌。
介绍公式时不妨也采用下面的方式,即先介绍公式的目的,给出公式,并逐一介绍公式中的变量。

\section{公式5.1与论证}
“从直接代码依赖的角度出发,从一个初始域外的类$C_{out}$ 出发我们尝试找到一个通往初始域内的类$C_{in}$ 的路径。一条合法的路径需要满足以下两点要求:(1)这一路径是单向的,即$C_{out}$ 传递性地到达$C_{in}$ 或$C_{in}$ 传递性地到达$C_{out}$;(2)路径中只能包含一个$C_{in}$ (为了避免重复路径的出现)。为了恰当的估计一条合法路径所代表的交互程度,我们计算路径上所有直接代码依赖的紧密度值的几何平均。我们用如下公式来重新计算给定$C_{out}$ 的IR 值($IR_{DC}$):”

\begin{align}
IR_{DC}=IR_{origin}+(IR_{top}-IR_{origin})^{\left| PATH\right|}\sqrt {\prod _{x \in PATH}Closeness_{DC}(x)} \end{align}

“其中$IR_{origin}$ 代表$C_{out}$ 的初始IR值,$IR_{top}$ 代表$C_{in}$ 被提升过的IR值,\emph{PATH} 代表$C_{out}$ 与$C_{in}$ 之间的路径内所有的直接代码依赖,而$Closeness_{DC}(x)$ 则代表每一条直接代码依赖关系的紧密度值。在同一对$C_{out}$ 和$C_{in}$ 之间可能存在多条合法路径,我们只保留其中能使$IR_{DC}$ 值最大的那条路径。”

\section{公式5.2与论证}

“由于IR 方法返回的是一个按照IR 值大小倒序排列的候选线索列表,因此一种常用的比较IR 方法的方式是在不同的查全率水平上比较不同方法之间的精确度,通常用$Precision-Recall$ 曲线表示。为了进一步衡量IR 方法返回结果的整体质量,我们选用了另外两个常用的实验度量:$AP$(Average Precision)与$MAP$ (Mean Average Precision)。其中,$AP$ 用于度量全部查询(需求)所检索的相关文档的排序质量,计算方式如下:”
\begin{align}
AP=\dfrac {\sum _{r=1}^{N}\left( Precision\left( r\right) \times isRelevant\left( r\right) \right) } {\left| RelevantDocuments\right| }
\end{align}

“其中,$r$ 表示被查询对象(类)在列表中的排序,$Precision(r)$ 表示前$r$ 个类的准确率。$isRelevant()$ 为一个二值函数,如果文档是相关的,则返回1,若无关,则返回0。”

\section{公式5.3与论证}

“由此,我们为类数据依赖定义紧密度$Closeness_{CD}$ 如下:”

\begin{align}Closeness_{CD}=\frac {\sum _{x \in \{DT_{i}\cap DT_{j}\}}idtf(x)} {\sum _{y \in \{DT_{i}\cup DT_{j}\}}idtf(y)}\end{align}

“其中$idtf(x)$ 代表共享数据类型的idtf值,$DT_i$ 与$DT_j$ 的交集代表该数据依赖上的共享数据类型,而$DT_i$ 与$DT_j$ 的并集则代表$C_i$ 和$C_j$ 在全部代码上所访问的数据类型。$Closeness_{CD}$ 的取值范围是0到1之间。”

\section{原模板中的其它公式}

\begin{equation}
\frac{\partial L}{\partial a_{k}^t} = {d(s)}^2 (y_{k}^t - \frac{\sum_{lab(\mathbf{l},k)} \alpha_t(s)\beta_t(s) }{y_{k}^t} )
\end{equation}

\begin{equation}
\begin{aligned}
d_{{0j}}&=\sum _{{k=1}}^{{j}}w_{{\mathrm  {ins}}}(a_{{k}}),\quad &{\text{for}}\;1\leq j\leq n\\
d_{{ij}}&={\begin{cases}d_{{i-1,j-1}}&{\text{for}}\;a_{{j}}=b_{{i}}\\\min {\begin{cases}d_{{i-1,j}}+w_{{\mathrm  {del}}}(b_{{i}})\\d_{{i,j-1}}+w_{{\mathrm  {ins}}}(a_{{j}})\\d_{{i-1,j-1}}+w_{{\mathrm  {sub}}}(a_{{j}},b_{{i}})\end{cases}}&{\text{for}}\;a_{{j}}\neq b_{{i}}\end{cases}}\quad &{\text{for}}\;1\leq i\leq m,1\leq j\leq n.
\end{aligned}
\end{equation}

\begin{equation}
\begin{aligned}
&\beta_T(|l{}'|)=y_{b}^{T}\\
&\beta_T(|l{}'|-1)=y_{l_|l|}^{T} \\
&\beta_T(s)=0, \forall s < |l{}'|-1
\end{aligned}
\end{equation}

递归公式举例(出现了公式过长的问题,实践中最好适当控制长度,给公式编号在同一行上留下位置)。
\begin{equation}
\beta_t(s)=\left\{
\begin{aligned}
& (\beta_{t+1}(s) d(s)+\beta_{t+1}(s+1))d(s+1)\,  y_{\l_s{}'}^t, \: \: if \:  l_s{}'=b \:  or \:  l_{s+2}{}'=l_s{}'\\
& (\beta_{t+1}(s) d(s)+\beta_{t+1}(s+1)d(s+1)+\beta_{t+1}(s+2)d(s+2))\,  y_{\l_s{}'}^t,\: \:   otherwise
\end{aligned}
\right.
\end{equation}
