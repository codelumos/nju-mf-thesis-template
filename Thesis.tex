% !Mode:: "TeX:UTF-8"
%# -*- coding:utf-8 -*-

%% 南京大学专业硕士学位论文的示例文档
%% 作者:https://github.com/codelumos
%% 模版repo:https://github.com/codelumos/nju-mf-thesis-template

\documentclass[winfonts,master,twoside,AutoFakeBold= {2}]{njuthesis}
%% 审阅模式很重要,命令是下方使用的\blind
%% njuthesis 文档类的可选参数有:
%%   winfonts, linuxfonts, macfonts, adobefonts winfonts 选项使得文档使用 Windows 系统提供的字体;linuxfonts 选项使得文档使用 Linux 系统提供的字体;macfonts 选项使得文档使用 Mac 系统提供的字体;adobefonts 选项使得文档使用 Adobe 提供的 OTF 中文字体(需自行下载安转)
%%   phd/master/bachelor 选择博士/硕士/学士论文
%%   twoside 或 oneside 指定排版的文档为双面打印或单面打印格式(twoside 会使得 chapter 章节从奇数页开始,即纸张的正面开始,因此会出现一些空白的页面)
%%   nobackinfo 取消封二页导师签名信息。注意,按照南大的规定,是需要签名页的。
%%   AutoFakeBold 设置CJK字体加粗,参数“2”用于指定加粗程度


%%%%%%%%%%%%%%%%%%%%%%%%%%%%%%%%%%%%%%%%%%%%%%%%%%%%%%%%%%%%%%%%%%%%%%%%%%%%%%%
% 格式化图表标题和子标题,详见:https://www.latexstudio.net/archives/8652.html
\captionsetup[subfigure]{labelformat=simple, labelsep=space}

%%%%%%%%%%%%%%%%%%%%%%%%%%%%%%%%%%%%%%%%%%%%%%%%%%%%%%%%%%%%%%%%%%%%%%%%%%%%%%%
% 设置《国家图书馆封面》的内容,仅博士论文才需要填写

% 设置论文按照《中国图书资料分类法》的分类编号
% \classification{0175.2}
% 该编号可在下述网址查询:https://udcsummary.info/php/index.php?lang=chi
% \udc{004.72}
% 国家图书馆封面上的论文标题第一行,不可换行。此属性可选,默认值为通过\title设置的标题。
% \nlctitlea{论文标题第一行}
% 国家图书馆封面上的论文标题第二行,不可换行。此属性可选,默认值为空白。
% \nlctitleb{论文标题第二行}
% 国家图书馆封面上的论文标题第三行,不可换行。此属性可选,默认值为空白。
% \nlctitlec{论文标题第三行}
% 导师的单位名称及地址
% \supervisorinfo{南京大学软件学院~~南京市汉口路22号~~210093}
% 答辩委员会主席
% \chairman{某某~~教授}
% 第一位评阅人
% \reviewera{某某~~教授}
% 第二位评阅人
% \reviewerb{某某~~副教授}
% 第三位评阅人
% \reviewerc{某某~~教授}
% 第四位评阅人
% \reviewerd{某某~~研究员}


%%%%%%%%%%%%%%%%%%%%%%%%%%%%%%%%%%%%%%%%%%%%%%%%%%%%%%%%%%%%%%%%%%%%%%%%%%%%%%%
% 设置论文的中文封面

% 单行论文标题,不可换行
\title{南京大学毕业论文\LaTeX 模板}

% 如果论文标题过长,可以分两行,第一行用\titlea{}定义,第二行用\titleb{}定义
% 使用以下3行:
% \title{} % 用于覆盖单行标题内容为空
% \titlea{长标题第一行}  % 第一行标题写这里
% \titleb{长标题第二行用于长标题换行} % 第二行标题写这里
% 注意:\title 不能都注释,它用于控制标题选择双行还是单行。\title{}如果内容为空,则编译\titlea{},\titleb{}双行标题,否则编译单行标题


% 论文作者姓名
\author{作者}
% 论文作者联系电话
\telphone{13012345678}
% 论文作者电子邮件地址
\email{author@smail.nju.edu.cn}
% 论文作者学生证号
\studentnum{MF20320000}
% 论文作者入学年份(级)
\grade{2020}
% 论文作者毕业年份(届), 出版授权书的学位年度
\graduateyear{2022}
% 导师姓名职称
\supervisor{某某~~教授,某某~~教授}
% 导师的联系电话
\supervisortelphone{}
% 论文作者的学科与专业方向
\major{工程硕士(软件工程领域)}
% 论文作者的研究方向
\researchfield{软件工程}
% 论文作者所在院系的中文名称
\department{软件学院}
% 论文作者所在学校或机构的名称。此属性可选,默认值为“南京大学”。
\institute{南京大学}
% 论文的提交日期,需设置年、月、日。
\submitdate{2022年5月20日}
% 论文的答辩日期,需设置年、月、日。
\defenddate{2022年5月20日}
% 论文的定稿日期,需设置年、月、日。
% 此属性可选,若注释\date{},则默认值为最后一次编译时的日期,精确到日。
% \date{2022年5月20日}

%%%%%%%%%%%%%%%%%%%%%%%%%%%%%%%%%%%%%%%%%%%%%%%%%%%%%%%%%%%%%%%%%%%%%%%%%%%%%%%
% 设置论文的英文封面

% 论文的英文标题,不可换行
\englishtitle{\LaTeX \;  NJU Thesis Template}
% 论文作者姓名的拼音
\englishauthor{Author}
% 导师姓名职称的英文
\englishsupervisor{Professor Alex, Professor Bob}
% 论文作者学科与专业的英文名
\englishmajor{Software Engineering}
% 论文作者所在院系的英文名称
\englishdepartment{Software Institute}
% 论文作者所在学校或机构的英文名称。此属性可选,默认值为“Nanjing University”。
\englishinstitute{Nanjing University}
% 论文完成日期的英文形式,它将出现在英文封面下方。需设置年、月、日。日期格式使用美国的日期
% 格式,即“Month day, year”,其中“Month”为月份的英文名全称,首字母大写;“day”为
% 该月中日期的阿拉伯数字表示;“year”为年份的四位阿拉伯数字表示。
% 此属性可选,若注释掉\englishdate{},则默认值为最后一次编译时的日期。
% \englishdate{May 20, 2022}

%%%%%%%%%%%%%%%%%%%%%%%%%%%%%%%%%%%%%%%%%%%%%%%%%%%%%%%%%%%%%%%%%%%%%%%%%%%%%%%
% 设置论文的中文摘要

% 设置中文摘要页面的论文标题及副标题的第一行。
% 此属性可选,其默认值为使用|\title|命令所设置的论文标题
% \abstracttitlea{标题第一行}
% 设置中文摘要页面的论文标题及副标题的第二行。
% 此属性可选,其默认值为空白
% \abstracttitleb{标题第二行用于长标题换行}

%%%%%%%%%%%%%%%%%%%%%%%%%%%%%%%%%%%%%%%%%%%%%%%%%%%%%%%%%%%%%%%%%%%%%%%%%%%%%%%
% 设置论文的英文摘要

% 设置英文摘要页面的论文标题及副标题的第一行。
% 此属性可选,其默认值为使用|\englishtitle|命令所设置的论文标题
% \englishabstracttitlea{englishabstracttitlea}
% 设置英文摘要页面的论文标题及副标题的第二行。
% 此属性可选,其默认值为空白
% \englishabstracttitleb{nglishabstracttitleb}

%%%%%%%%%%%%%%%%%%%%%%%%%%%%%%%%%%%%%%%%%%%%%%%%%%%%%%%%%%%%%%%%%%%%%%%%%%%%%%
%% 盲审命令,空白字段设置请看 .cls文件 \newcommand*{\blind}
%% 此外,请按照盲审要求自行去掉个人简历、致谢等页面中的个人信息
%%**********************!!!非常重要的盲审命令,送审前必选!!!****************
% \blind
%%**********************!!!非常重要的盲审命令,送审前必选!!!****************

%%%%%%%%%%%%%%%%%%%%%%%%%%%%%%%%%%%%%%%%%%%%%%%%%%%%%%%%%%%%%%%%%%%%%%%%%%%%%%%
\begin{document}

%%%%%%%%%%%%%%%%%%%%%%%%%%%%%%%%%%%%%%%%%%%%%%%%%%%%%%%%%%%%%%%%%%%%%%%%%%%%%%%

% 制作国家图书馆封面(博士学位论文才需要)
% \makenlctitle
% 制作中文封面
\maketitle
% 制作英文封面
\makeenglishtitle


%%%%%%%%%%%%%%%%%%%%%%%%%%%%%%%%%%%%%%%%%%%%%%%%%%%%%%%%%%%%%%%%%%%%%%%%%%%%%%%
% 开始前言部分
\frontmatter

\begin{abstract}
	这部分是中文摘要。
	
	\textit{注意:本模板使用的是XeLaTeX编译的,这一编译的好处在于通过支持utf-8编码格式直接支持中文。}
	
	模板与使用指南基本信息如下:
	\begin{itemize}
		\item 本模板参考了多份出自于自计算机系Haixing Hu提供的基于XeLaTeX编译的本科LaTeX模板。感谢之前各位同学和老师的贡献!
		\item 本模板已由学院内多位老师指正,确保满足毕业论文要求,使用过程中如发现模板存在的问题请及时反馈到khy@nju.edu.cn。
		\item 本模板内的文字内容与使用指南由软件学院的匡宏宇完成,改编自学院之前为研究生提供的硕士毕业论文模板,保留了各个参考模板提供的一些表格、图形和算法例子。输出的pdf文件内的内容均可以在tex文件中找到对应,尽可能地方便各位同学改编使用。
		\item 推荐使用TeXLive作为编译器(类似于JDK),TeXStudio作为编辑器(类似于Eclipse),二者均为开源软件且支持三大主流操作系统。环境安装与配置请参考这条知乎专栏 \footnote{\url{https://zhuanlan.zhihu.com/p/80603542}}。注意,在Thesis.tex文件开头有针对不同环境的字库选项,一定要选择适合自己操作系统的字体,默认为Windows。
		\item  出于严谨性的考量,本模板与使用指南目前仅对学院专业硕士开放(学硕如果要使用需要略作修改,可邮件联系我),请大家不要外传。
	\end{itemize}
	
	常见问题与解答
	\begin{enumerate}
		\item LaTeX模板并不与Word模板完全一致,也无需与Word模板的格式一致,使用本模板即遵照本模板的要求。
		\item 编写论文时建议拷贝一份本模板pdf文件的副本,原模板内的文字是模板使用说明,以及常用LaTeX格式的用法(表格、图形、论文引用、公式、算法等),方便各位同学参考。模板中留下的注释也提供了一些使用方面的指导,请多加留意。
		\item LaTeX已经非常成熟,通过搜索引擎可以解决绝大部分问题,模板内也提供了丰富的样例和额外的手册供参考。
		\item 每位同学可以选择自己搭建编译器与编辑器的组合,但如果不了解LaTeX的话建议还是使用推荐配置。
	\end{enumerate}
	% 同时应该注意到,空白页是故意留白,以便章节开头能够出现在偶数页
	% 中文关键词,关键词之间用中文全角分号隔开,末尾无标点符号
	\keywords{手写中文;文本识别;深度学习}
\end{abstract}

%%%%%%%%%%%%%%%%%%%%%%%%%%%%%%%%%%%%%%%%%%%%%%%%%%%%%%%%%%%%%%%%%%%%%%%%%%%%%%%
% 论文的英文摘要
\begin{englishabstract}
	The official sites of tools mentioned in this template:
	
	\begin{itemize}
		\item TeXLive: \url{https://tug.org/texlive/}
		\item TeXStudio: \url{https://www.texstudio.org/}
		\item SumatraPDF: \url{https://www.sumatrapdfreader.org/}
	\end{itemize}

	The guidelines of how to install these tools can be found in corresponded sections of this template.
	% 英文关键词,关键词之间用英文半角逗号隔开,末尾无符号
	\englishkeywords{Handwritten Chinese, Text recognition, Deep learning}
\end{englishabstract}

%%%%%%%%%%%%%%%%%%%%%%%%%%%%%%%%%%%%%%%%%%%%%%%%%%%%%%%%%%%%%%%%%%%%%%%%%%%%%%%
% 论文的前言,应放在目录之前,中英文摘要之后,一般不需要
%
% \begin{preface}
%
% 在过去的40年中,手写中文文本领域识别(HCTR)取得了很大的进展[1,2]。
%
% \vspace{1cm}
% \begin{flushright}
% 某某某\\
% 2022年5月20日于南大鼓楼
% \end{flushright}
%
% \end{preface}

%%%%%%%%%%%%%%%%%%%%%%%%%%%%%%%%%%%%%%%%%%%%%%%%%%%%%%%%%%%%%%%%%%%%%%%%%%%%%%%
% 生成论文目录
\tableofcontents

%%%%%%%%%%%%%%%%%%%%%%%%%%%%%%%%%%%%%%%%%%%%%%%%%%%%%%%%%%%%%%%%%%%%%%%%%%%%%%%
% 生成插图清单,如无需插图清单则可注释掉下述语句
\listoffigures

%%%%%%%%%%%%%%%%%%%%%%%%%%%%%%%%%%%%%%%%%%%%%%%%%%%%%%%%%%%%%%%%%%%%%%%%%%%%%%%
% 生成附表清单,如无需附表清单则可注释掉下述语句
\listoftables

%%%%%%%%%%%%%%%%%%%%%%%%%%%%%%%%%%%%%%%%%%%%%%%%%%%%%%%%%%%%%%%%%%%%%%%%%%%%%%%
% 开始正文部分
\mainmatter

%%%%%%%%%%%%%%%%%%%%%%%%%%%%%%%%%%%%%%%%%%%%%%%%%%%%%%%%%%%%%%%%%%%%%%%%%%%%%%%
% 学位论文的正文应以《绪论》作为第一章,本模板是按照自身功能模块组织的,并非论文中的章节安排

\input{chapter/Chapter1_Title.tex}

\input{chapter/Chapter2_MainBody.tex}

\input{chapter/Chapter3_Table.tex}

\input{chapter/Chapter4_Figure.tex}

\input{chapter/Chapter5_Formula.tex}

\input{chapter/Chapter6_Algorithm.tex}

\input{chapter/Chapter7_Reference.tex}

%%%%%%%%%%%%%%%%%%%%%%%%%%%%%%%%%%%%%%%%%%%%%%%%%%%%%%%%%%%%%%%%%%%%%%%%%%%%%%%
% 致谢,应放在结论之后
\begin{acknowledgement}
	感谢在实验室度过的两年时光,老师无论在学术还是人生的指导上都对我起到了很大的帮助;师兄师姐小伙伴们的鼓励支持和陪伴是我坚持下去的动力。
\end{acknowledgement}




% 参考文献,应放在\backmatter之前
% 推荐使用BibTeX,若不使用BibTeX时注释掉下面一句
% \nocite{*}
\bibliography{references}


% 附录,必须放在参考文献后,backmatter前
\appendix
\chapter{附录代码}\label{app:1}
\section{main函数}
\begin{lstlisting}[language=C]
int main()
{
	return 0;
}
\end{lstlisting}

%%%%%%%%%%%%%%%%%%%%%%%%%%%%%%%%%%%%%%%%%%%%%%%%%%%%%%%%%%%%%%%%%%%%%%%%%%%%%%%
% 书籍附件
\backmatter
%%%%%%%%%%%%%%%%%%%%%%%%%%%%%%%%%%%%%%%%%%%%%%%%%%%%%%%%%%%%%%%%%%%%%%%%%%%%%%%
% 作者简历与科研成果页,应放在backmatter之后
\begin{resume}
% 论文作者身份简介,一句话即可
\begin{authorinfo}
\noindent 某某某,男,汉族,1997年11月出生,江苏省泰州人。
\end{authorinfo}
% 论文作者教育经历列表,按日期从近到远排列,不包括将要申请的学位
\begin{education}
\item[2016年9月 --- 2020年6月] 南京大学软件学院 \hfill 本科
\end{education}
% 论文作者在攻读学位期间所发表的文章的列表,按发表日期从近到远排列
\begin{publications}
\item Wenbo Yang, Wentao Zhu, ``Voting-on-Grid Clustering for Secure
  Localization in Wireless Sensor Networks,'' in \textsl{Proc. IEEE International
    Conference on Communications (ICC) 2010}, May. 2010.
\item Wenbo Yang, Wentao Zhu, ``Protecting Source Location Privacy
  in Wireless Sensor Networks with Data Aggregation,'' in \textsl{Proc. 6th
    International Conference on Ubiquitous Intelligence and Computing (UIC)
    2019}, Oct. 2019.
\end{publications}
% 论文作者在攻读学位期间参与的科研课题的列表,按照日期从近到远排列
\begin{projects}
\item 国家自然科学基金面上项目“问题研究”
(课题年限~2010年1月 --- 2012年12月),负责相关问题的研究。
\end{projects}
\end{resume}

%%%%%%%%%%%%%%%%%%%%%%%%%%%%%%%%%%%%%%%%%%%%%%%%%%%%%%%%%%%%%%%%%%%%%%%%%%%%%%%
% 生成《学位论文出版授权书》页面,应放在最后一页
% \makelicense

%%%%%%%%%%%%%%%%%%%%%%%%%%%%%%%%%%%%%%%%%%%%%%%%%%%%%%%%%%%%%%%%%%%%%%%%%%%%%%%
\end{document}
